\documentclass[10pt]{report} 

\begin{document}

\subsection{Document}
document class, begin and end are all macros that take arguments inside curly brackets
The documentclass[]{}commanddefines thedocumentclassdocumentclass, where[]contains optional arguments such as font size.
The classis chosen between{}. Standard classes arearticle,book,report,slides, andletter


\subsection{Environment}
Environments contain special content, such as math, figures, tables, etc. Environmentsstart with begin{}and end with end{}, where the environment name is between{}.Thedocumentenvironment is most important: all content within this environment will beprinted


\subsection{Sectioning}
We can use section, subsection, sub subsection and paragraph{}  to create sections.
% paragraph name

\subsection{bold, italic and underlined}
\textbf{bold} \\ 
\textit{italic} \\ 
\underline{underlined}

There are a few special characters including \&, \%, \$, \{ \} for {} and \#.

\subsection{Cross references}
We may refer to a section, subsection, math equation or figure by using label.


% page layout = geometry package

%not  everything we need comes with latex installation so similar to having packages or modules in codes, we use packages to load them
%using use package command


% math section

% figures

% tables

% references


% sections

% enumerate and items (lists)

% \includegraphics


% setting margins

% setting font size

% picking a font



%spacing between lines


% turning page number on/off


%footnotes

%alignment


%colored text


% compilation

% errors



\end{document}





